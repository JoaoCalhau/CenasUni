%%%%%%%%%%%%%%%%%%%%%%%%%%%%%%%%%%%%%%%%%
% Beamer Presentation
% LaTeX Template
% Version 1.0 (10/11/12)
%
% This template has been downloaded from:
% http://www.LaTeXTemplates.com
%
% License:
% CC BY-NC-SA 3.0 (http://creativecommons.org/licenses/by-nc-sa/3.0/)
%
%%%%%%%%%%%%%%%%%%%%%%%%%%%%%%%%%%%%%%%%%

%----------------------------------------------------------------------------------------
%	PACKAGES AND THEMES
%----------------------------------------------------------------------------------------

\documentclass{beamer}

\mode<presentation> {

% The Beamer class comes with a number of default slide themes
% which change the colors and layouts of slides. Below this is a list
% of all the themes, uncomment each in turn to see what they look like.

%\usetheme{default}
%\usetheme{AnnArbor}
%\usetheme{Antibes}
%\usetheme{Bergen}
%\usetheme{Berkeley}
%\usetheme{Berlin}
%\usetheme{Boadilla}
%\usetheme{CambridgeUS}
%\usetheme{Copenhagen}
%\usetheme{Darmstadt}
%\usetheme{Dresden}
%\usetheme{Frankfurt}
%\usetheme{Goettingen}
%\usetheme{Hannover}
%\usetheme{Ilmenau}
%\usetheme{JuanLesPins}
%\usetheme{Luebeck}
\usetheme{Madrid}
%\usetheme{Malmoe}
%\usetheme{Marburg}
%\usetheme{Montpellier}
%\usetheme{PaloAlto}
%\usetheme{Pittsburgh}
%\usetheme{Rochester}
%\usetheme{Singapore}
%\usetheme{Szeged}
%\usetheme{Warsaw}

% As well as themes, the Beamer class has a number of color themes
% for any slide theme. Uncomment each of these in turn to see how it
% changes the colors of your current slide theme.

%\usecolortheme{albatross}
%\usecolortheme{beaver}
%\usecolortheme{beetle}
%\usecolortheme{crane}
%\usecolortheme{dolphin}
%\usecolortheme{dove}
%\usecolortheme{fly}
%\usecolortheme{lily}
%\usecolortheme{orchid}
%\usecolortheme{rose}
%\usecolortheme{seagull}
%\usecolortheme{seahorse}
%\usecolortheme{whale}
%\usecolortheme{wolverine}

%\setbeamertemplate{footline} % To remove the footer line in all slides uncomment this line
%\setbeamertemplate{footline}[page number] % To replace the footer line in all slides with a simple slide count uncomment this line

%\setbeamertemplate{navigation symbols}{} % To remove the navigation symbols from the bottom of all slides uncomment this line
}

\usepackage{graphicx} % Allows including images
\usepackage{booktabs} % Allows the use of \toprule, \midrule and \bottomrule in tables
\usepackage[utf8]{inputenc}

%----------------------------------------------------------------------------------------
%	TITLE PAGE
%----------------------------------------------------------------------------------------

\title[]{IoT Eye Tracking in Elderly Care} % The short title appears at the bottom of every slide, the full title is only on the title page

\author{João Calhau} % Your name
\institute[UE] % Your institution as it will appear on the bottom of every slide, may be shorthand to save space
{
University of Évora \\ % Your institution for the title page
\medskip
\textit{m36764@alunos.uevora.pt} % Your email address
}
\date{\today} % Date, can be changed to a custom date

\begin{document}

\begin{frame}
\titlepage % Print the title page as the first slide
\end{frame}

\begin{frame}
\frametitle{Overview} % Table of contents slide, comment this block out to remove it
\tableofcontents % Throughout your presentation, if you choose to use \section{} and \subsection{} commands, these will automatically be printed on this slide as an overview of your presentation
\end{frame}

%----------------------------------------------------------------------------------------
%	PRESENTATION SLIDES
%----------------------------------------------------------------------------------------

%------------------------------------------------
\section{Internet of Things} % Sections can be created in order to organize your presentation into discrete blocks, all sections and subsections are automatically printed in the table of contents as an overview of the talk

\begin{frame}
\frametitle{Internet of Things}
Internet of Things is a technological revolution that consists in connecting objects that we use everyday to a global network, the Internet. Internet of Thing's idea is to transform the digital and physical world in one, through devices that can communicate with each other.
\end{frame}

%------------------------------------------------

\subsection{Applications} % A subsection can be created just before a set of slides with a common theme to further break down your presentation into chunks

%------------------------------------------------

\begin{frame}
\frametitle{Applications}
\begin{itemize}
\item Connected Industry: "things" connected between themselves in the industry area , like printers, cranes or whole mines.
\item Smart Cities: Intelligent cities that aim to create sustainability conditions, improve the way of life of the population through the analysis of data recollected from cities. 
\item Smart Energy: Intelligent electric networks that make use of information technology to make the system more efficient, both economically and energetically.
\end{itemize}
\end{frame}

%------------------------------------------------

\section{Eye Tracking}

\begin{frame}
\frametitle{Eye Tracking}
Thinking of all the times we can use our eyes during a day, we can quickly realize that Eye Tracking as tons of applications possible, and as it embraces so many activities in our day-to-day life we can also realize that the areas to which we can apply this technology are also varied. With very few exceptions, some thing that has a visual component can be tracked. Eye tracking can be used in the automotive and medical areas to make our lives safer, the entertainment and web design areas has already benefited significantly by studding the way our visual behavior. \\ 
Everyday, as Eye Tracking is used, it is used in new and creative ways and because of that it's application list grows everyday.
\end{frame}

%------------------------------------------------

\section{Elderly Care}

\begin{frame}
\frametitle{Elderly Care}
Having talked about IoT and Eye Tracking we can now talk about these two topics focused in a single area, the Elder care area. \\
Elderly care is the fulfillment of the special needs and requirements that are unique to senior citizens. This broad term encompasses such services as assisted living, adult day care, long term care, among others.
\end{frame}

%------------------------------------------------
\subsection{Applications}
%------------------------------------------------

\begin{frame}
\frametitle{Applications}
\begin{block}{Interactive TVs}
Interactive TVs are one of the possible applications to Eye Tracking technology as they have a particularly interesting aspect that would benefit senior citizens tremendously, the access to a lot of services from home. \\
Some of which could be information research, viewing habits customization or even shopping.
\end{block}
\begin{block}{Driving}
It's never safe when a senior citizen drives a car, with age reaction times are slower, vision  and audition clarity as well as muscle strength and flexibility are lost, some drowsiness begins to show because of certain types of medicine and there seems to be a lack of ability to concentrate. With Automotive Eye Tracking all that changes, it could detect drowsiness or distraction and warn the driver, it could allow a certain degree of personalization and it could allow the car to identify it's driver and change it's settings according to the user's preference.
\end{block}
\end{frame}

%------------------------------------------------

\begin{frame}
\frametitle{Applications (cont.)}
\begin{block}{Medicine}
With Eye Tracking technology it could even be possible to detect if an elderly person has, or not, Parkinson's disease through a non invasive method. The theorized test uses infrared light to track the movements of the eye while the person looks at a screen and follows the prompts. \\
Another use for Eye Tracking could be the diagnose of Alzheimer's disease. The only current ways to determine if a person has, or not, Alzheimer's is through a series of detailed and invasive medical tests, with Eye Tracking there could be a new, non evasive way, to test for Alzheimer's. The only thing the patient would have to do is to look at images on a screen while a camera tracks the movements of his eyes. 
\end{block}
\end{frame}

%------------------------------------------------

\begin{frame}
\Huge{\centerline{The End}}
\end{frame}

%----------------------------------------------------------------------------------------

\end{document}