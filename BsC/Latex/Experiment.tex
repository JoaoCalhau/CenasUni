\documentclass[a4paper,12pt]{article} %inicialização do tipo de documento
\usepackage{graphicx} %package para as imagens
\usepackage[utf8]{inputenc} %package para a linguagem
\usepackage[T1]{fontenc} %outra package para linguagem
\usepackage[portuguese]{babel} %package para a linguagem portuguesa
\author{João Calhau - 31621,\\ André Figueira - 31626} %definir os autores do domumento
\title{Experiment} %definir o titulo
\date{\today}
\begin{document} %começar o documento
\maketitle %meter o titulo (que engloba o titulo (claro), os autores e a data)
\tableofcontents %fazer a tabela de contents (tudo o que o documento tem)

\newpage

\section{Some Interesting Words} %criar uma secçao para a tabela de contents
Well, and here begins my lovely article
\newline %mudar de linha
(Don't forget to check here for things you dont remember :D )

\newpage %mudar de pagina

\section{Good Bye World} %outra secçao

Here's some stuff \ldots{}

\subsection{Here's part of it} %sub-secção

If one merely wishes to type in ordinary text, without
complicated mathematical formulae or special effects such
as font changes, then one merely has to type it in as it
is, leaving a completely blank line between successive
paragraphs.

\subsubsection{Here's the rest of it} %sub-secção da sub-secção (Subception O.o)

You do not have to worry about paragraph indentation:
all paragraphs will be indented with the exception of
the first paragraph of a new section.

One must take care to distinguish between the `left quote'
and the `right quote' on the computer terminal.  Also, one
should use two `single quote' characters in succession if
one requires ``double quotes''.  One should never use the
(undirected) `double quote' character on the computer
terminal, since the computer is unable to tell whether it
is a `left quote' or a `right quote'.  One also has to
take care with dashes: a single dash is used for
hyphenation, whereas three dashes in succession are required
to produce a dash of the sort used for punctuation---such as
the one used in this sentence.

\newpage %nova pagina
\section{Retards} %outra secçao

Well we all know retards \ldots{} It's true!

\begin{figure}[ht!] %começar uma imagem (jpg)
\centering %centrar imagem
\includegraphics[width=90mm]{birds.jpg} %definicoes de imagem e imagem em si
\caption{You know it!} %meter capcoes na imagem (aparece por baixo, ao lado de "figura 1: "
\label{overflow}
\end{figure} %terminar edicao de imagem

\ldots{} and here it ends.

\end{document} %terminar documento